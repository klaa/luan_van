{\begin{center}
{\textbf MỞ ĐẦU}
\end{center}}
\addcontentsline{toc}{chapter}{Mở đầu}
% \centerline{\bf \large\MakeUppercase{LỜI NÓI ĐẦU}}
\vspace{10pt}
Theo báo cáo của Tổ chức Y tế thế giới (TCYTTG - WHO Report 2020 - Global Tuberculosis Control)\cite{gtcreport} ước tính năm 2019 trên toàn cầu có khoảng 10 triệu người hiện mắc lao. Bệnh lao là nguyên nhân gây tử vong đứng hàng thứ hai trong các bệnh nhiễm trùng với khoảng 1,2 triệu người tử vong do lao.

Về Việt nam, hiện chúng ta vẫn là nước có gánh nặng bệnh lao cao, đứng thứ 11 trong 30 nước có số người bệnh lao cao nhất trên toàn cầu, đồng thời đứng thứ 11 trong số 30 nước có gánh nặng bệnh lao kháng đa thuốc cao nhất thế giới\cite{gtcreport}. Hàng năm, ước tính có 17.000 trường hợp tử vong do lao tại Việt Nam.

Chẩn đoán bệnh lao không thật sự khó trong đa số các trường hợp. Điều đáng chú ý là làm sao chẩn đoán sớm và chẩn đoán đúng để khởi động điều trị sớm nhằm giảm các tổn thương cũng như biến chứng của lao gây ra. Để làm được điều trên, việc ứng dụng công nghệ thông tin vào quá trình chuẩn đoán là thực sự cần thiết, đặc biệt là áp dụng những tiến bộ của học máy, học sâu để xây dựng lên hệ thống hỗ trợ chuẩn đoán bệnh lao.

Xuất phát từ sự cần thiết của việc áp dụng những tiến bộ học máy, học sâu như đã trình bày, học viên đã thực hiện đề tài "\tenluanvan". Đề tài đảm bảo được sự phù hợp và tính khoa học cần thiết, đặc biệt có tính ứng dụng thực tiễn cao. 

Mặc dù đã có cố gắng nỗ lực, song luận văn không tránh khỏi những thiếu sót do năng lực và thời gian hạn chế. Em chân thành mong muốn lắng nghe những đóng góp, góp ý của thầy, cô, bạn bè, đồng nghiệp để luận văn được cải thiện tốt hơn.

Em xin chân thành cảm ơn.
