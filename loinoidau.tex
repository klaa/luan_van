\chapter*{ \begin{center}
MỞ ĐẦU
\end{center}}
\addcontentsline{toc}{chapter}{Mở đầu}
% \centerline{\bf \large\MakeUppercase{LỜI NÓI ĐẦU}}
\vspace{10pt}
Theo báo cáo của Tổ chức Y tế thế giới (TCYTTG - WHO Report 2020 - Global Tuberculosis Control) \cite{gtcreport}, mặc dù đã đạt được một số thành tựu đáng kể trong công tác chống lao trong thời gian qua, bệnh lao vẫn đang tiếp tục là một trong các vấn đề sức khoẻ cộng đồng chính trên toàn cầu. TCYTTG ước tính năm 2019 trên toàn cầu có khoảng 10 triệu người hiện mắc lao, một con số đã giảm rất chậm trong những năm gần đây; 8,2\% trong số mắc lao có đồng nhiễm HIV. Bệnh lao là nguyên nhân gây tử vong đứng hàng thứ hai trong các bệnh nhiễm trùng với khoảng 1,2 triệu người tử vong do lao và khoảng 208.000 người chết do lao trong số những người nhiễm HIV. Số tử vong này làm cho lao là một trong các bệnh gây tử vong hàng đầu ở nữ giới. WHO đã công bố kết quả của mô hình đánh giá tác động ngắn hạn của đại dịch Covid-19 lên số ca tử vong do lao trong năm 2020. Kết quả cho thấy rằng tử vong do lao có thể tăng đáng kể trong năm 2020 và sẽ ảnh hưởng đến nhóm bệnh nhân lao dễ bị tổn thương nhất, tăng khoảng 200.000 – 400.000 ca tử vong, nếu như các dịch vụ chẩn đoán và phát hiện bệnh nhân trên toàn cầu bị ngưng trệ và giảm từ 25 – 50\% trong khoảng 3 tháng. Con số tử vong sẽ tương ứng với mức tử vong toàn cầu do lao vào năm 2015, một bước lùi nghiêm trọng trong quá trình hướng tới mục tiêu của Hội nghị Cấp Cao Liên Hợp Quốc về Lao và Chiến lược thanh toán bệnh lao của WHO.

Về Việt nam, hiện chúng ta vẫn là nước có gánh nặng bệnh lao cao, đứng thứ 11 trong 30 nước có số người bệnh lao cao nhất trên toàn cầu, đồng thời đứng thứ 11 trong số 30 nước có gánh nặng bệnh lao kháng đa thuốc cao nhất thế giới (báo cáo WHO 2020). Bệnh lao vẫn là một trong những bệnh truyền nhiễm phổ biến ở Việt Nam. Hàng năm, ước tính có 17.000 trường hợp tử vong do lao tại Việt Nam, cao hơn gấp hai lần so với con số tử vong do tai nạn giao thông. Mỗi năm ước tính có 180.000 người có bệnh lao hoạt động; 5.000 trường hợp trong số đó được xác định nhiễm lao kháng đa thuốc. Chẩn đoán bệnh lao không thật sự khó trong đa số các trường hợp. Điều đáng chú ý là làm sao chẩn đoán sớm và chẩn đoán đúng để khởi động điều trị sớm nhằm giảm các tổn thương cũng như biến chứng của lao gây ra. Để làm được điều trên, việc ứng dụng công nghệ thông tin vào quá trình chuẩn đoán là thực sự cần thiết, đặc biệt là áp dụng những tiến bộ của học máy để xây dựng lên hệ thống hỗ trợ chuẩn đoán bệnh lao.

\textbf{Đối tượng nghiên cứu:} Ảnh X-quang lồng ngực trong y tế thu nhận bởi các máy chiếu, chụp chuyên dụng.

\textbf{Phạm vi nghiên cứu:} Ảnh đa mức xám chụp phổi thẳng thường quy ( tư thế sau - trước), chụp phổi nghiêng thường quy và chụp đỉnh phổi tư thế ưỡn ngực.

\textbf{Những nội dung nghiên cứu chính:} Dự kiến nội dung báo cáo của luận văn gồm: phần mở đầu, 3 chương chính, phần kết luận, tài liệu tham khảo, phụ lục. Bố cục được trình bày như sau:

\begin{enumerate}[wide, label=\bfseries Chương \arabic*:]
	\item [\bfseries Phần mở đầu:] Nêu lý do chọn đề tài và hướng nghiên cứu chính
	\item \tenchuongi 
	\item \tenchuongii
	\item \tenchuongiii
\end{enumerate}

Mặc dù đã có cố gắng nỗ lực, song luận văn không tránh khỏi những thiếu sót do năng lực và thời gian hạn chế. Em chân thành mong muốn lắng nghe những đóng góp, góp ý của thầy, cô, bạn bè, đồng nghiệp để luận văn được cải thiện tốt hơn.

Em xin chân thành cảm ơn.
