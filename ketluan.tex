\chapter*{\begin{center}Kết luận\end{center}}
\addcontentsline{toc}{chapter}{Kết luận}
Phân loại ảnh số là một lĩnh vực nghiên cứu hấp dẫn vì có thể áp dụng trong rất nhiều bài toán thực tế. Đây cũng là một bài toán phức tạp nhưng không quá khó để giải quyết nếu ta biết ứng dụng các thành tựu nghiên cứu trong các lĩnh vực như xử lý ảnh số, trí tuệ nhân tạo… Trong đó, việc ứng dụng thành quả của Deep learning mà trong đó đặc biệt là các mô hình của mạng CNN cho ta các kết quả thực sự ấn tượng.

\section*{Kết quả đã đạt được}
Sau một thời gian tìm hiểu nghiên cứu, luận văn đã trình bày được các vấn đề sau:
\begin{itemize}
	\item Trình bày khái quát về CNN và bài toán chẩn đoán bệnh lao
	\item Hệ thống hóa một số mô hình học sâu hỗ trợ chẩn đoán
	\item Cài đặt thử nghiệm một trong các mô hình đã được hệ thống hóa
\end{itemize}

\section*{Hướng hoàn thiện và phát triển tiếp theo:}
Chương trình tuy đã đảm bảo được những chức năng chính yếu nhất của luận văn, nhưng để áp dụng vào thực tế thì vẫn chưa thể được. Lý do chính cho việc này là do sự khác biệt giữa nguồn ảnh đầu vào. Như đã trình bày ở Chương 3, nguồn ảnh đầu vào của bài toán luận văn là ảnh định dạng PNG, tuy nhiên, thực tế nguồn ảnh x-quang y tế được chụp qua các thiết bị thu nhận ảnh y tế (CT, MRI...) hầu hết lại ở định dạng DICOM. Việc không đồng nhất về định dạng ảnh khiến cho chương trình hiện nay chưa thể đưa vào sử dụng trong thực tế.

Một vấn đề khác khi nghiên cứu bài toán của luận văn với bộ dữ liệu do Tawsifur Rahman và cộng sự \cite{dataset} cung cấp là chất lượng ảnh đầu vào không đồng đều. Hầu hết ảnh trong bộ dữ liệu đều có chất lượng tốt, sắc nét, rõ ràng. Nhưng cũng có một vài ảnh mờ, không thực sự rõ nét. Ảnh chất lưởng kém hơn ít nhiều cũng sẽ ảnh hưởng đến độ chính xác của chương trình. Việc nâng cao chất lượng ảnh đầu vào cho chương trình nói riêng và cả lĩnh vực Thị giác Máy - Computer Vision - nói chung là vấn đề rất quan trọng. 

Từ những vấn đề nêu trên, tác giả đề xuất hướng phát triển tiếp theo là hoàn thiện thêm các chức năng liên quan đến nâng cao chất lượng ảnh đầu vào, chức năng kết nối với thiết bị thu nhận ảnh y tế (CT, MRI...) để hoàn thiện chương trình có thể ứng dụng vào thực tiễn.